\chapter{Úvod}
\pagenumbering{arabic}
\setcounter{page}{1}
Kontejnery pro virtualizaci existují již řadu let. Rozšíření kontejnerů přišlo před čtyřmi lety s firmou Docker a jejich technologiemi. Hlavní rozdíly mezi kontejnery a klasickou virtualizací jsou ve spotřebě zdrojů. \newline
Pro virtualizaci založenou na kontejnerech je klíčové sdílení systémových zdrojů s hostitelským operačním systémem, oproti klasické, která celý systém ve virtuálním stroji emuluje a vytváří tak vlastní zdroje.

Práce s dnešními kontejnerizačními nástroji je velmi jednoduchá a intuitivní. Během krátké doby začaly používat kontejnerové technologie ve svých infrastrukturách firmy jako Spotify, Yahoo či Google\cite{docker_part}. Můžeme předpokládat, že s rozšiřováním cloudových služeb se bude také zvyšovat zapojování kontejnerů.

V dnešní době rychlého škálování a růstu firem je velmi důležitá dynamika, pomocí které je možné kontejnery spravovat.

Práce je zaměřená na prozkoumání problematiky kontejnerové virtualizace. Cílem práce je převedení aplikace a zjištění, jaký orchestrátor je pro produkční kontejnerové prostředí optimální.

Obsah bakalářské práce je rozčleněn do 6 kapitol. V následující kapitole je popsán princip kontejnerů a porovnání kontejnerových technologií. V kapitole 3 je vysvětlen princip orchestrátoru a porovnání jednotlivých typů. Kapitola 4 se zaměřuje na analýzu migrované aplikace a výběr vhodných nástrojů pro migraci. Migrace této aplikace je pak  popsaná v kapitole 5. Závěrečná kapitola 6 je věnována shrnutí výsledků praktické práce a doporučením.

V bakalářské práci se vyskytují anglické výrazy. Výrazy, které nemají české ekvivalenty, nebudou v této práci přeloženy tak,u jak je v technických publikacích běžné.
