\appendix
\pagenumbering{Roman}

\chapter{Cizí výrazy}
\begin{itemize}
\item cgroups, namespaces - Linuxové zdroje zabudované v kernelu, pomocí který se izolují kontejnerové technologie od hostitelského systému.
\item VM(Virtual Machine) - Virtuální stroj - Jedná se o produkt virtualizace, jedná se o software, který emuluje hardware a dokáže tak spustit operační systém, který má obdobné vlastnosti jako operační systém nainstalovaný na hardwaru.
\item Hardware - Fyzicky existujicí technické vybavení počítače.
\item Open source - Otevřený kód, progamy označovaný open source má volně přístupné zdrojové kódy většinou se tyto kódy nachází na serveru GitHub.
\item Linoxový daemon - Program, který je spuštěn dlouhodobě, není v přímém kontakt s uživatelem
\item Hypervisor - Hostitelský operační systém, který řídí přístup k virtualizovaným systémům
\item Runtime - Knihovny potřebné k spustění aplikace
\item Microslužby - Je tip architektury, která staví svůj koncept na rozdělení služeb nezávislých celků. 
\item Git - je SCM nástroj na správu verzi Softwarů, autorem je Linus Torvalds (autor Linuxové kernelu) 
\item Linuxový kernel - Monolithické jádro, které ovládá a spravuje hardwarové zdroje, jedná se zároveň na největší open source projekt. 
\item GitHub - Největší veřejný repozitář pro ukládání kódu v systému gitu.
\item Single point of failure - Je část aplikace, která při své nedostupnosti může způsobit výpadky celé aplikace.
\item Cluster - Jedná se o skupinu propojených služeb/počítačů, které navzájem spolupracují
\item init - Proces, který je spuštěn v unixových systémech jako první.
\item Scheduler - Technologie, které slouží k plánování a spouštěním procesů na jednou. 
\end{itemize}
