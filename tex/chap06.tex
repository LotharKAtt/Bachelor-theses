\chapter{Závěry a doporučení}
\section{Závěr}
Kontejnerová virtualizace je zajisté evolucí v moderním světě virtualizace, ve kterém se nyní firmy odklánějí od proprietárních virtualizačních technologií k open source řešením. Kontejnery se svým  přístupem a prací se zdroji mění celkový pohled na virtualizaci, jsou nesrovnatelně rychlejší než klasická virtualizace.

Tato práce si dala za cíl prověřit, zda je možné převést do kontejnerů klasickou aplikaci. V práci byla zmíněna problematika kontejnerové virtualizace. Pomocí vybraných nástrojů byla převedena jednoduchá webová aplikace do kontejnerového prostředí. Na této aplikaci pak bylo demonstrováno několik případů užití orchestrátoru. Z výsledku lze usoudit, že převedení aplikace do kontejnerového prostředí je možné, záleží však na architektuře aplikace a použití dostupných technologií. Využití kontejnerové virtualizace ve standardním podnikovém prostředí bez použití orchestrátoru nedává smysl. 

\section{Budoucnost kontejnerových technologií}
Lze téměř s jistotou tvrdit, že historie se opakovat nebude a kontejnery se už stanou plnohodnotným virtualizačním řešením, které se bude běžně používat v produkčních prostředích. Díky OCi a komunitě okolo alternativních runtime je předpoklad,že  dojde k odklonu od Dockeru, protože jako rkt či cri-o pracuje spolehlivě.

Orchestrátory čeká zajisté podobná cesta jako projekt OpenStack. Velké firmy začínají vydávat své vlastní distribuce, které implementují lepší podporu pro jejich stávající produkty a služby. Kubernetes se už tak dočkal své distribuce od Canonicalu a CoreOS. Firma Redhat si nad Kubernetes postavila celou kontejnerovou platformu OpenShift. Kontejnerová virtualizace se bude zajisté i nadále měnit a rozvíjet.